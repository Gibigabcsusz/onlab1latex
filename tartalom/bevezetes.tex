\chapter{Bevezetés}
Ebben a dolgozatban a BME VIK Villamosmérnök MSc képzés Önálló Laboratórium 1 c. tárgyának keretében végzett kutatási és tervezési munkámat összegzem. A dolgozatom témája egy kevéssé ismert nyomtatott antennatípus, a BIFA (Back Inverted F Antenna) tervezése.
\section{Kitűzött feladatok}
\section{BIFA}
\section{Céges háttér}
A BIFA betűszó feloldása: Back Inverted F Antenna. Ez az elnevezés nem gyakori a szakirodalomban, elsősorban a Silicon Laboratories cég kontextusában lehet vele találkozni. Ez az antennatípus egy variációja az IFA-nak (Inverted F Antenna), ezért az IFA jellegzetességeiből kiindulva érdemes tárgyalni.
\par
\begin{center}
\textit{IFA kép}
\end{center}
\par
Az IFA alapvetően egy 
\section{Céges háttér}
